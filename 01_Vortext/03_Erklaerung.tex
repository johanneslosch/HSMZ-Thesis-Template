% Erklärender Text zu dieser Datei --------------------------------------------------------
% Die Datei 03_Erklaerung.tex dient der Definition der Eidesstattlichen Erklärung.
% Die Datei enthält den Text, der im Leitfaden der Hochschule Mainz vorgegeben ist.
% Wenn ein Bild einer Unterschrift eingefügt werden soll, ist an der Stelle %TODO dem Hinweis zu folgen.
% -----------------------------------------------------------------------------------------
\pdfbookmark{Erkl"arung}{erklaerung}
\addchap*{Eigentständigkeitserklärung}
Hiermit versichere ich, dass ich die anliegende Arbeit selbst angefertigt und alle für die Arbeit verwendeten Quellen
und Hilfsmittel in der Arbeit vollständig angegeben habe. Zudem versichere ich, dass ich weder diese noch inhaltlich verwandte Arbeiten als Prüfungsleistung in anderen Fächern eingereicht habe oder einreichen werde.
\\
\textbf{Ich versichere au{\ss}erdem, (eine der folgenden Optionen ist mit dem / der Prüfer:in abzustimmen!):}


\setlist[itemize]{leftmargin=*}
\begin{itemize}
    \item[\rlap{\raisebox{0.3ex}{\hspace{0.4ex}\scriptsize \ding{56}}}$\square$] Option 1: ßErlaubnis von KI-Tools mit geringfügiger Dokumentation\\ ass ich die Nutzung aller für die Erstellung dieser Arbeit erlaubten generativen KI-Tools im Anhang der vorliegenden Arbeit durch die Benennung des Tools und seines Einsatzzwecks tabellarisch dokumentiert habe. Alle wortwörtlich übernommenen oder paraphrasierten Inhalte von generativen KI-Tools wurden entsprechend gekennzeichnet. Mir ist bewusst, dass der Versuch einer nicht-dokumentierten Nutzung generativer KI-Tools als Täuschungsversuch entsprechend der Prüfungsordnung zu werten ist. Ich versichere, dass ich die mithilfe von KI-Tools generierten Inhalte nicht unreflektiert übernommen habe und ich als Autor:in die Verantwortung für Angaben und Aussagen in dieser Arbeit trage.
    \item[$\square$] Option 2: Erlaubnis von KI-Tools mit umfangreicher Dokumentation\\ dass ich die Nutzung aller für die Erstellung dieser Arbeit erlaubten generativen KI-Tools im Anhang der vorliegenden Arbeit tabellarisch dokumentiert habe. Dazu gehört, welche generativen KI-Tools ich für welchen Zweck verwendet habe und auf welche Ausschnitte der Arbeit sich diese Nutzung bezieht. Ich habe zudem alle, von dem / der Prüfer:in angeforderten Quellen (Prompts \& Outputs), die meinen Einsatz von generativer KI-Tools bei der Erstellung dieser Arbeit nachweisen, nach bestem Wissen im Anhang der Arbeit zur Verfügung gestellt. Alle wortwörtlich übernommenen oder paraphrasierten Inhalte von generativen KI-Tools wurden entsprechend gekennzeichnet. Mir ist bewusst, dass der Versuch einer nicht-dokumentierten Nutzung generativer KI-Tools als Täuschungsversuch entsprechend der Prüfungsordnung zu werten ist. Ich versichere, dass ich die mithilfe von KI-Tools generierten Inhalte nicht unreflektiert übernommen habe und ich als Autor:in die Verantwortung für Angaben und Aussagen in dieser Arbeit trage.
    \item[$\square$] Option 3: dass ich \textit{keine} auf Künstlicher Intelligenz (KI) basierenden Text- oder Bildgeneratoren (z.B. ChatGPT) verwendet habe, da der Einsatz dieser Tools bei der Erstellung dieser Arbeit mir durch den / die Prüfer:in explizit verboten wurde. Mir ist bewusst, dass der Einsatz eines generativen KI-Tools eine Missachtung der Vorgabe ist und dies als Täuschungsversuch entsprechend der Prüfungsordnung gewertet wird.
\end{itemize}
\vspace{3em}
%TODO Wenn eine Unterschrift eingefügt werden soll, muss nachfolgender Kommentar eingefügt werden
% Einfuegen der Unterschrift als Datei "signature.png", Scale muss ggf. angepasst werden
% \begin{figure}[h]
%     \hspace{9cm}
%     \includegraphics[scale=0.2]{04_Artefakte/01_Abbildungen/signature.png}
% \end{figure}
\ort, den \datumAbgabe
\vspace{-1cm} % notwendig, dass ort und datum inc erster Name auf gleicher Höhe sind
\begin{flushright}
    \autor
\end{flushright}

